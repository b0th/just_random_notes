\documentclass{article}
\title{Just random notes}
\author{b0th}

\usepackage{geometry}
\geometry{legalpaper, margin=1.2in}

\usepackage{listings}
\usepackage{xcolor}

\definecolor{codegreen}{rgb}{0,0.6,0}
\definecolor{codegray}{rgb}{0.5,0.5,0.5}
\definecolor{codepurple}{rgb}{0.58,0,0.82}
\definecolor{backcolour}{rgb}{0.95,0.95,0.92}

\lstdefinestyle{mystyle}{
    backgroundcolor=\color{backcolour},   
    commentstyle=\color{codegreen},
    keywordstyle=\color{codepurple},
    numberstyle=\tiny\color{codegray},
    stringstyle=\color{codepurple},
    basicstyle=\ttfamily\footnotesize,
    breakatwhitespace=false,         
    breaklines=true,                 
    captionpos=b,                    
    keepspaces=true,                                
    showspaces=false,                
    showstringspaces=false,
    showtabs=false,                  
    tabsize=2
}

\lstset{style=mystyle}

\begin{document}
\maketitle

\begin{description}
    \item[Runtime] \mbox{}\\Runtime describes software/instructions that are executed while your program is running, especially those instructions that you did not write explicitly, but are necessary for the proper execution of your code.
    \item[KISS] \mbox{\textit{Keep It simple, stupid}}\\ The KISS principle states that most systems work best if they are kept simple rather than made complicated.
    \item[Container] \mbox{}\\ Group of namespaces and control groups applied to a process.
    \item[Linux kernel namespace] \mbox{}\\ Limit what the process sees, here some namespaces
    
    \begin{itemize}
        \item item
        \item pid
        \item net
        \item mnt
        \item uts
        \item ipc
        \item user
    \end{itemize}

    C functions to manage them

    \begin{itemize}
        \item clone()
        \item unshare()
    \end{itemize}

    \item[Linux kernel cgroup] \mbox{\textit{Control group}}\\ Limit what the process can use, here some cgroups
    \begin{itemize}
        \item memomry
        \item CPU
        \item network
        \item devices
        \item pids
    \end{itemize} 

    \item[C++ inheritance class] \mbox{}\\ Single inheritance
\begin{lstlisting}[language=C++]
class Rectangle: public Shape {
    public:
       int getArea() { return (width * height); }
};
\end{lstlisting}
    Multiple inheritance
\begin{lstlisting}[language=C++]
class Rectangle: public Shape1, Shape2, Shape3 {
    public:
       int getArea() { return (width * height); }
};
\end{lstlisting}
    \item[C++ namespace] \mbox{}\\ Namespaces allow to group entities like classes, objects and functions under a name. Example of declaration
\begin{lstlisting}[language=C++]
namespace myNamespace
{
  int a = 0;
}
\end{lstlisting}

    Usage

\begin{lstlisting}[language=C++]
std::cout << myNamespace::a << std::endl
\end{lstlisting}
    or
\begin{lstlisting}[language=C++]
using namespace myNamespace;
std::cout << a << std::endl
\end{lstlisting}

    \item[C++ cout] \mbox{\textit{character out}}\\
    \item[C++ endl] \mbox{\textit{end line}}\\ 
    \item[Makefile special variables] \mbox{}\\
      
\begin{lstlisting}
all: library.cpp main.cpp
\end{lstlisting}

\begin{lstlisting}
$@ evaluates to all 
$< evaluates to library.cpp
$^ evaluates to library.cpp main.cpp
\end{lstlisting}

    \item[Web CGI] \mbox{\textit{Common Gateway Interface}}\\
    Set of standards that define how information is exchanged between the web server and a custom script.

\end{description}
\end{document}